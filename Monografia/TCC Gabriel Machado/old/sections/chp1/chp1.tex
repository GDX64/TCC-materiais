
\section{Análise Espectral}

Um dos conceitos mais antigos em análise de sinais é o de frequência. As funções senoidais, ou as exponencias complexas de modo geral, são fascinantes por possuírem diversas propriedades matemáticas interessantes, como por exemplo serem solução para diversas equações diferencias e quando são entrada de um sistema linear invariante no tempo, este sistema apresenta como saída um sinal da mesma classe da entrada, ou seja, outra exponencial complexa. O conceito de frequência sequer precisa estar ligado à series temporais. Em imagens ele também está presente na forma de frequência espacial. O próprio termo 'espectro' aparentemente foi introduzido por Newton enquanto estudava a decomposição da luz em diferentes cores. Tudo isso faz com que a decomposição de sinais em frequência seja um dos tópicos mais estudados em processamento de sinais. \cite{stoica2005spectral} \cite{castanie2013digital}.

É quase impossível dissociar o tópico da famosa transformada de Fourier, mas olhando da perspectiva do que é basicamente uma TF há diversas outras bases sobre as quais se pode projetar um sinal qualquer de modo a extrair determinadas características. Também é fato de que hoje esta análise se dá na maioria dos casos por meios digitais, dado o poder computacional que se tem à disposição atualmente e a fundamentação teórica sólida que dão teoremas como o famoso Teorema da amostragem de Nyquist-Shannon \cite{mitra2006digital} \cite{lago2002digital}. Ao final estas são as técnicas que se utiliza basicamente neste trabalho, todo ele será baseado em análise digital de sinais e sua maior parte com técnicas relacionadas à TF.

\section{Aplicações}

As aplicações da análise espectral são as mais diversas. Em processamento de imagens temos seu uso para encontrar fronteiras e comprimir arquivos, reduzindo seu tamanho e economizando banda de transmissão e espaço de armazenamento \cite{baxes1994digital}. 

Em processamento de fala, se pode utilizar para extrair características de fonemas, de modo a posteriormente classificá-los \cite{huang2001spoken}.

Em biomedicina, se pode usar para análise de eletrocardiogramas, e eletroencefalogramas na detecção de ondas alpha, beta, gamma e outras, que indicam determinadas atividades cerebrais e podem ser detectadas por análise espectral \cite{sornmo2005bioelectrical}.  

Além das demais citadas há uma área em especial que vem ganhando cada vez mais atenção e à qual se dedicam as simulações dos métodos que serão apresentados neste trabalho: a de qualidade de energia elétrica. Com o aumento constante de dispositivos não lineares ligados à rede elétrica, monitorar e filtrar harmônicos indesejados se torna indispensável. Atualmente existem diversas fontes de energia e características de cargas. A grande interligação que existe entre os sistemas de potência também traz preocupação de grandes quedas no suprimento. Em resumo, o sistema de distribuição de energia elétrica está aumentando em complexidade, e desta maneira se fazem necessárias técnicas mais arrojadas para monitorá-lo e ajudar em seu controle \cite{dugan1996electrical}.

Observando atentamente o leitor se dará conta de que as simulações são em sua grande maioria feitas considerando uma componente fundamental em 60 Hz e principalmente os harmônicos ímpares até o 15º, que são os que normalmente aparecem em sistemas de potência. O trabalho se baseia nos seguintes artigos \cite{de2009pll} e \cite{chang2009two}, onde ambos os métodos propostos são concebidos com o intuito de analisar harmônicos e inter-harmônicos presentes comumente na rede elétrica.

\section{Divisão}

O capítulo 2 trata de uma fundamentação teórica e revisão bibliográfica (como indica o título) traçando os fundamentos da estimação espectral e conceitos de processamento de sinais. Neste capítulo discorre-se sobre a teoria por trás dos dois métodos mencionados anteriormente que são base deste trabalho, e também estão presentes algumas demonstrações relevantes para o entendimento dos mesmos. 

O capítulo 3 fala especificamente do método PLL Multitaxa, suas características, os problemas que este visa solucionar e porque foi concebido desta maneira, mais alguns conceitos específicos que possam ter ficado de fora do capítulo 2. Estão incluídos resultados de simulações realizadas em MATLAB, na forma de figuras e tabelas.

O capítulo 4 trata da parte de estimação de frequências do método Adaline de dois estágios, e segue na mesma estrutura do capítulo 3: apresentamos sua construção, pontos fortes, fracos e resultados de simulação.

O capítulo 5 encerra o trabalho com um método unindo os dois anteriormente apresentados de forma complementar, utilizando o estágio de estimação de frequências presente no capítulo 4 para inicializar o método PLL Multitaxa. Ao final também são feitas algumas ponderações e proposições para trabalhos futuros.