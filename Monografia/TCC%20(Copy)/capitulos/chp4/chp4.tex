\documentclass[a4paper, 12pt]{book}

\usepackage[portuguese]{babel}
\usepackage{listings}
\usepackage{graphicx}
\usepackage{calc}
\usepackage{amsmath}
\usepackage{bigints}
\usepackage{tikz}
\usetikzlibrary{shapes.geometric, shapes,arrows}
\usepackage{xcolor}
\usepackage{float}


\begin{document}

\section{Identificação de frequências}

\indent Vimos na seção sobre o PLL-Multitaxa que este é um método com grande eficácia no rastreio de frequências, entretanto precisamos saber onde elas estão, e isto o método não nos diz. Seria possível por exemplo utilizá-lo como um filtro de partículas e largar PLLs aleatoriamente pelo espectro do sinal, entretanto temos algumas formas de obter boas ideias de onde estão as componentes mais relevantes do sinal. Uma delas é a DFT ou a STFT com janelamento para reduzir o espalhamento pelo espectro, desta forma podemos identificar regiões do espectro onde sabemos que temos energia relevante, sem saber no entanto a localização exata da componente, ou as componentes as quais pertence esta energia. Outro método visto na revisão foi o de Prony, do qual se fez uso no trabalho [3]. Neste trabalho utilizaram da otimização via RLS para resolver o problema da predição linear e calcularam as raízes do polinômio característico do modelo Auto-Regressivo do sinal, assim puderam obter as componentes relevantes com certa precisão.

\indent Neste mesmo trabalho, posteriormente se faz uso do RLS novamente para estimar a amplitude e fase de cada uma das componentes, alimentando o RLS com sinais senoidais em quadratura de modo que o filtro estimado fosse a amplitude de senoides e cossenoides [obs: Vou colocar uma diagrama disso depois]. 

\indent Nesta seção faremos a exposição de um método baseado no trabalho [3] com algumas modificações.

\subsection{Solução em predição linear}

O método planteado na seção de Análise de Prony visa encontrar os coeficientes $w_m$ tais que:

\begin{equation}
u[k]=\sum_{m=1}^{M}w_m u[k-m]
\end{equation}

Que ao final vimos que é o equivalente a encontrar a matrize de autocorrelação e o vetor de correlação cruzada do sinal $u[k]$ na forma:

\begin{equation}
\boldsymbol{w}_{opt}=\boldsymbol{R}_{uu}^{-1}\boldsymbol{r}_{du}
\end{equation}

Onde desta vez o sinal de referência $d[k]$ ganha um caráter especial uma vez que é $u[k+1]$. Estas equação pode ser resolvida utilizando o método de Levinson–Durbin, para um determinado conjunto de dados. Mas também pode ser resolvida com os filtros adaptativos a cada iteração, fazendo um rastreio online destes parâmetros.

\subsection{Solução em RLS e NLMS}

Como vimos anteriormente, o RLS faz uma estimação da matriz $\boldsymbol{R}_{uu}^{-1}$ a cada iteração e geralmente é o método que nos vai dar menor MSEe em menos tempo, entretanto ele tem algumas deficiências:

\begin{itemize}
	\item Se a matriz $\boldsymbol{R}_{uu}$ é singular ou não é bem ajustada, podemos ter problemas numéricos significativos, como a ordem dos elementos da matriz ser muito desbalanceada ou grande demais. Isso é bastante notável em nosso problema de estudo, pois em geral não sabemos a ordem do sistema que estamos analisando, se este sistema tem uma ordem pequena e o estimamos com um filtro grande, a matriz $\boldsymbol{R}_{uu}$ certamente será singular. Algo que podemos fazer para minimizar este problema é a adição de ruído.
	\item Em geral é mais lento para reagir a variações no sistema que o LMS e NLMS.
	\item Tem uma complexidade computacional consideravelmente maior.
	\item Sua convergência é de certa forma caótica, perturbando o cálculo das raízes.
\end{itemize}

Devemos também destacar que tanto o RLS quanto o NLMS se beneficiam de ordens de filtro próximas a quantidade de amostras por ciclo da fundamental. Isto se deve ao fato de que quanto menor são as variações dos sinais dentro do buffer $\boldsymbol{U}$ mais singular é a matriz de autocorrelação. 

\begin{figure}[h]
	\centering    
	\def\svgwidth{\columnwidth}
	\input{convergencia_RLS_NLMS.pdf_tex}
	\caption{Convergência dos coeficientes RLS e NLMS na presença dos harmônicos 1, 3, 5 e 7; M=16}
	\label{fig:your image label}
\end{figure}

Fazendo uma simulação com os harmônicos de 1 até 7, com a amplitude igual ao inverso de sua ordem e ruído gaussiano com $\sigma=0.02$, comparamos os resultados da divisão do menor pelo maior autovalor da matriz de autocorrelação estimada do sinal com diferentes valores de amostragem e ordem da matriz:

\begin{table}[H]
	\centering
	\begin{tabular}{l|l|l}
		   & M=16 & M=64 \\
		\hline 
		f0x16      & 1.3e-03 & 2.1e-04 \\
		f0x64      & 2.7e-04  & 8.6e-05       
	\end{tabular}
\end{table}

\begin{figure}[h]
	\centering    
	\def\svgwidth{\columnwidth}
	\input{erro_RLS_NLMS.pdf_tex}
	\caption{Convergência do RLS e NLMS na presença dos harmônicos 1, 3, 5 e 7; M=16}
	\label{fig:your image label}
\end{figure}

\end{document}