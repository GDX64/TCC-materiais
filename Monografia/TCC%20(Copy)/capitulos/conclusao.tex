\section{Conclusões}
Neste trabalho apresentamos o que é e o problema que representa hoje a estimação espectral, bem como a fundamentação teórica matemática por trás do tema. Discorremos sobre métodos paramétricos e não paramétricos. Nos focamos principalmente em simular e apresentar alternativas para duas metodologias relativamente recentes, publicadas nos últimos 10 anos. 

Apresentamos as virtudes e pontos fracos de cada um dos métodos, ao final do trabalho é proposto um método capaz de unir os pontos fortes de cada um. Aproveitando a robustez e simplicidade computacional no rastreio do PLL-M e o poder de identificação global dos métodos baseados em modelos AR. Entretanto, mesmo dentro da metodologia aplicada ao capítulo final, há muito o que se desenvolver. Um método mais robusto seria capaz de identificar o surgimento de novas componentes e o desaparecimento de outras, fazendo uma administração inteligente dos recursos computacionais.

Mesmo a estimação baseada em modelos Auto Regressivo tem fortes limitações, como já foi discutido. Se consideramos um sinal determinístico que está altamente poluído por ruído, não é possível fazer uma boa estimação com este modelo. 

De um modo geral podemos listar as boas condições de trabalho do método proposto:

\begin{itemize}
	\item Soma de sinais senoidais finitos menores que M. 
	\item Relação sinal ruído maior que 20 dB.
	\item A ordem do modelo em si não é um agravante, mas a forma como estão distribuídas as senoides, sim pode ser. Os testes mostram que temos pouquíssimos erros para frequências separadas em no mínimo 0.04 radianos por segundo.
\end{itemize}

Por Parte do PLL temos as seguintes condições:
\begin{itemize}
	\item O algoritmo é capaz de rastrear componentes mesmo em relações sinal ruído de 0 dB.
	\item Degraus em frequência por volta de 5 Hz em qualquer componente. 
	\item Sinais dentro de certas amplitudes as quais as constantes setadas são capazes de seguir.
\end{itemize}

\section{Trabalhos futuros}

Para futuras melhorias estão:
\begin{itemize}
	\item Determinar quando o primeiro estágio convergiu e já estimou corretamente as frequências e ordem do modelo. O que não é trivial, uma vez que pode ser difícil saber quando o algoritmo atingiu o erro mínimo. O RLS por exemplo é um método de mínima variância, capaz de convergir para o menor erro possível, que seria o de ruído. Entretanto como não se sabe o ruído, isto não é tão simples. E tampouco é trivial determinar as frequências corretas e as falsas.
	\item Estudar métodos paramétricos baseados em sinais determinísticos poluídos por ruído. Assim não teremos as limitações de uma aproximação por modelos AR ou ARMA. 
	\item Encontrar um método sistemático de obtenção das constantes do PLL e outras formas de analisar sua convergência e dinâmica. Uma vez que é um método altamente não linear e complexo de se analisar. Nos trabalhos analisados foram usadas diversas soluções heurísticas que carecem de mais fundamento matemático, como a junção do PLL com um filtro IIR.
\end{itemize}