% para incluir imagens:
% inkscape -D -z --file=simple_pll_harmonicos.svg --export-pdf=simple_pll_harmonicos.pdf --export-latex
	
Como visto na seção referente ao PLL, esse método é capaz de minimizar o erro quadrático médio entre um sinal $y(t)$ e uma componente senoidal para ao menos um mínimo local. Neste capítulo ganhará forma uma estrutura reunindo o PLL e o processamento multitaxa com o objetivo de reduzir a complexidade computacional e conseguir um algoritmo mais versátil. 

\section{Desenpenho do PLL}
Nas figuras seguintes pode-se observar como converge o algoritmo em diferentes situações, todas foram simuladas para um sinal de 180 V de amplitude e constantes 300, 500, e 6, com frequência de amostragem igual a 7680 Hz, partindo de condições inicias $f_i=60 Hz$ e $A=0$. Percebe-se pela simulação que o algoritmo converge rapidamente mesmo com ruído, entretanto, na presença de harmônicos com a mesma quantidade de energia, a convergência já é bastante comprometida. Ainda assim, em valor médio, a estimação se mostra correta. Conclui-se que é possível estimar harmônicos diretamente com o algoritmo PLL obtido, entretanto é conveniente aliá-lo a outras técnicas para que se diminua o erro da estimação. Os resultados das simulações podem ser vistos nas figuras \ref{fig:PLL_conv1} e \ref{fig:PLL_conv2}.

\begin{figure}[h]
	\centering    
	\def\svgwidth{\columnwidth}
	\input{images/simple_pll_harmonicos.pdf_tex}
	\caption{Convergência na presença do 3º e 5º harmônicos; SNR=10 dB}
	\label{fig:PLL_conv1}
\end{figure}

\begin{figure}[h]
	\centering    
	\def\svgwidth{\columnwidth}
	\input{images/pll_ruido.pdf_tex}
	\caption{Convergência na presença de ruído; SNR=10 dB}
	\label{fig:PLL_conv2}
\end{figure}

\section{banco de filtros}

\indent A solução encontrada em \cite{de2009pll} para parte dos problemas anteriormente citados é o uso de um preprocessamento com filtros passa-banda, para melhorar a relação sinal ruído, e posteriormente subamostragem, para diminuir a complexidade computacional, de modo também que não seja necessário mudar as constantes do algoritmo.

\indent O conjunto de filtros utilizado é uma cascata de dois filtros IIR com a seguinte função de transferência:

\begin{equation}
H_{bp}(z)=\frac{1-\alpha}{2}\frac{1-z^{-2}}{1-\beta (1-\alpha) z^{-1} + \alpha z^{-2}}
\label{eq:filtro}
\end{equation}
\begin{equation}
\beta=cos(w_0)
\label{eq:w0 filtro}
\end{equation}

\indent O parâmetro $\alpha$ modifica a seletividade do filtro e está entre 0 e 1, para que este seja estável. Quanto mais próximo de 1, mais seletivo é o filtro. O parâmetro $\beta$ modifica a frequência central do filtro de acordo com a equação \ref{eq:w0 filtro}, onde $w_0$ é a frequência normalizada de acordo com a amostragem.

\indent Este filtro é uma boa escolha por alguns motivos:
\begin{itemize}
	\item Ele rechaça completamente a componente DC do sinal.
	\item Tem atraso de fase nulo na frequência central.
	\item É paramétrico, suas características dependem dos parâmetros $\alpha$ e $\beta$, os quais modificam propriedades muito específicas do filtro, sendo entAõ muito fácil utilizá-lo e modificá-lo em tempo real.
\end{itemize}

\begin{figure}[h]
	\centering    
	\def\svgwidth{\columnwidth}
	\input{images/banco_de_filtros.pdf_tex}
	\caption{Características do filtro com $w_0$=0.25}
	\label{fig:your image label}
\end{figure}

\indent Tem-se talvez como única desvantagem o atraso de grupo que é máximo para a frequência central, e quanto mais seletivo é o filtro, maior é esse atraso. Deve-se também ter atenção com a frequência de amostragem, pois quanto maior, menos seletivo um mesmo $\alpha$ seria. Se por exemplo, se mantém $\alpha$ e aumenta $f_s$, frequências que antes estavam normalizadas mais longe de nossa frequência central anterior, agora estariam mais próximas, por assim dizer. Dessa forma quanto maior é $f_s$ mais seletivo deve ser o filtro para a separação das mesmas frequências. Isso acaba se tornando um jogo de balanceamento destes parâmetros, pois como vimos anteriormente, um $\alpha$ maior também eleva o atraso de grupo, entretanto se tem mais amostras em menos tempo devido ao aumento em $f_s$. Todos estes efeitos estão muito bem descritos por J. Rodrigues em seu trabalho \cite{carvalho2008estimaccao}. Ao final utilizaremos $\alpha = 0.975$ e dois filtros em cascata para o restante das simulações.

\section{Uso da estrutura multitaxa}

Depois de passado pelo filtro passa-banda, pode-se realizar a subamostragem do sinal, uma vez que em tese eliminamos os harmônicos mais distantes quase que completamente, e estes não interferirão na estimação. Dentro revisão bibliográfica foi discutido o perfil desta interferência e também como encontrar a posição de uma frequência depois de esta sofrer aliasing. Agora é feita a exposição do algoritmo em C de uma função simples capaz de calcular esta frequência, que pode ser até mais explicativo que o texto anterior:

\lstinputlisting[language=c]{capitulos/chp3/algoritmo.c}

A função recebe como argumentos a frequência de amostragem Fs, a frequência rastreada \text{f\_init}, e o ponteiro para a variável flag, que indica se a frequência encontrada estava no semicírculo positivo, quando vale 1, ou negativo, quando vale -1 ($\theta<\pi$ ou $\theta>\pi$). É importante guardar esta informação porque necessitamos dela para recuperar a frequência original estimada. 

\begin{figure}[h]
	\centering    
	\def\svgwidth{\columnwidth}
	\input{images/circulo_freq.pdf_tex}
	\caption{círculo de frequências}
	\label{fig:freq_circ}
\end{figure}

\indent É possível observar na figura \ref{fig:freq_circ} a localização de duas frequências f1 e f2 no círculo. Suponhamos que a frequência de amostragem é $f_s=240 \:Hz$, $f_2=\frac{240}{2} \frac{3}{4} \: Hz = 90 \: Hz$ e $f_2=\frac{240}{2} \frac{4}{3} \: Hz = 160 \: Hz$, encontramos facilmente suas colocações no círculo utilizando o algoritmo citado. Agora reparemos que diferentes frequências serão mapeadas nas mesmas posições do círculo, por exemplo $f_1=\frac{240}{2} \frac{11}{4} \: Hz = 330 \: Hz$ também é mapeada no mesmo ângulo, então não existe uma função inversa que devolva as frequências mapeadas para seus valores reais. 

\indent Também é importante se atentar ao fato de que quando se aumenta $f_1$ levemente, se está aumentando sua frequência aparente, mas quando se faz o mesmo com $f_2$ se está diminuindo sua frequência aparente. Por isso é importante guardar a variável 'flag', ela diz se acréscimos positivos em frequência aparente condizem à acréscimos ou decréscimos na frequência real, e uma vez que não existe uma função inversa que nos devolva a frequência real dada a aparente, deve-se utilizar da variação na estimação aparente e o valor inicial mapeado para recuperar a estimação real.

\indent Uma coisa que se deve ter em mente é que algumas frequências se localizarão muito próximas a zero, e terão frequência aparente muito pequena. Isso dificulta muito a análise, é desejável testar a localização antes de iniciar o algoritmo e fazer uma mudança no valor de subamostragem caso necessário.

\begin{equation}
\hat{f}=f_{init}+\Delta f_{aparente}\, flag
\end{equation}

\section{variação da frequência central do banco de filtros}

\begin{figure}[h]
	\centering    
%	\def\svgwidth{\columnwidth}
	\def\svgscale{0.7}
	\input{images/esquema_pll.pdf_tex}
	\caption{esquema multitaxa PLL}
	\label{fig:esquema_pll}
\end{figure}

\indent A estrutura multitaxa consiste em passar o sinal de entrada $x(t)$ pelo banco de filtros, sub-amostrar e passar este sinal para os respectivos PLLs. Como o filtro utilizado é bastante seletivo, a frequência estimada deve controlar o banco de filtros centrando a frequência corretamente. A maneira mais básica de realizar este controle seria alimentar o banco diretamente com as frequências obtida no PLL, entretanto este método é inviável. Vimos na seção sobre o algoritmo PLL utilizado que este é altamente não linear, e complexo de se analisar a convergência. Fazer uma realimentação deste tipo muda o sistema e pode torná-lo instável, ou prejudicar sua convergência. A alternativa encontrada por J. Rodrigues \cite{carvalho2008estimaccao}  é utilizar a média das últimas $L$ estimações, assim o filtro passa-banda se move de maneira mais suave e não prejudica tanto o PLL. Desta maneira:
   
\begin{equation}
w_0[k]=\frac{1}{L} \sum_{i=k-L+1}^{k}\hat{w}[i]
\end{equation}

\indent Uma outra opção é fazer a estimação com uma série geométrica, ou soma convexa, que pode ser calculada de forma recursiva:
 
\begin{equation}
w_{0}[k]=w_{0}[k-1](1-\lambda) + \lambda \hat{w}[k]
\label{eq:w0 filtro}
\end{equation}

Onde $\lambda$ é uma constante entre 0 e 1. \\

\begin{figure}[h]
	\centering    
	%	\def\svgwidth{\columnwidth}
	\def\svgscale{1}
	\input{images/media_direto.pdf_tex}
	\caption{Comparação entre o método de média (com 24 amostras) e o de alimentação direta}
	\label{fig:esquema_pll}
\end{figure}

\begin{figure}[h]
	\centering    
	%	\def\svgwidth{\columnwidth}
	\def\svgscale{1}
	\input{images/geometrico_media.pdf_tex}
	\caption{Comparação entre o método de média (com 24 amostras) e o geométrico ($\lambda=0.9$)}
	\label{fig:esquema_pll}
\end{figure}

\indent Simulando as três opções, a alimentação direta da estimação realmente se mostra mais instável, e com maior tempo de convergência, enquanto que o método geométrico em geral é levemente mais estável e de mais rápida convergência que o de média, além de ser mais fácil de computar. Todos foram testados para as mesmas constantes do exemplo anterior, amplitude de 180 V e 60 Hz iniciais, a constante de subamostragem escolhida foi $M_k=16$. Depois de 1 segundo de simulação, é aplicado um degrau na frequência de -2 Hz. Também é passado um filtro média móvel de 16 amostras nos resultados finais de $\hat{w}$ e $\hat{A}$.

\section{Síntese da estrutura PLL Multitaxa}

\begin{enumerate}
	\item Passamos o sinal por um banco de filtros.
	\item Abaixamos a frequência em $M_k$ amostras.
	\item Calculamos a frequência aparente $f_a$ da componente a rastrear.
	\item Inicializamos o PLL com esta $f_a$ e a guardamos para recompor a original.
	\item Fazemos as estimações de $w$ e $A$, usamos a equação \ref{eq:w0 filtro} e utilizamos a série geométrica para realimentar o banco de filtros com a frequência calculada.
\end{enumerate}

\section{simulações}

Para as simulações computacionais foram seguidos determinados critérios:

\begin{itemize}
	\item Convergência em amplitude: Foram consideradas as últimas 32 estimações, assim que a média do erro quadrático destas fosse menor que 4e-4 (que é um erro menor que 2\% do sinal), consideramos que o algoritmo convergiu.
	\item O mesmo para a convergência em amplitude, mas com umbral de 1e-4.
	\item Os cálculos de erro quadrático médio são feitos considerando o valor das estimações depois de 1 segundo. Quando o algoritmo já convergiu.
\end{itemize}

Os parâmetros da simulação são $f0=60Hz$, 128 pontos por ciclo de $f0$, $Fs=7380 \,Hz$, $\alpha=0.975$ e dois filtros em cascata no banco. A constante de subamostragem é 16, para todos os harmônicos e inter-harmônicos. A amplitude da fundamental é 180 V, e para todos os harmônicos suas amplitudes são a da fundamental dividido pela ordem do respectivo harmônico. Foi passado um filtro média móvel de ordem 16 em todas as estimações de modo a suavizar a visualização dos resultados.

\subsection{harmônicos ímpares}

\indent Abaixo estão os resultados para a simulação com os harmônicos ímpares de 1 a 15. Verificamos algumas coisas:
\begin{itemize}
	\item o valor de offset para a fundamental é relativamente grande, e isto se explica principalmente pela subamostragem, uma vez que alguns harmônicos não são completamente atenuados se sobrepõe à fundamental. Isso poderia ser melhorado com o aumento da seletividade do filtro, entretanto já discutimos os problemas que isto pode ocasionar. Seu uso ou não depende da natureza do problema em questão.
	\item o erro em frequência é em geral mais baixo que o de amplitude. E também é seu tempo de convergência.
	\item o tempo de convergência é difícil de prever, não é certo que os harmônicos de ordem mais elevada ou os de maior energia convergirão mais rapidamente que os demais.
	\item o algoritmo se comporta muito bem com ruído, principalmente a estimação de frequência.
\end{itemize} 
\begin{table}[H]
	\begin{tabular}{|p{2.5cm}|p{2.5cm}|p{2.5cm}|p{2.5cm}|p{2.5cm}|}
		\hline
		Harmônico & Erro em frequência (\%) & Erro em amplitude (\%) &
		Tempo de convergência W (s) & tempo de convergência Amp. (s)\\
		\hline
		1  & 6,14E-07 & 1,344 & 0,144 & 0,098 \\
		3  & 2,08E-06 & 1,427 & 0,150 & 0,135 \\
		5  & 2,16E-06 & 1,338 & 0,035 & 0,088 \\
		7  & 3,05E-05 & 1,441 & 0,090 & 0,194 \\
		9  & 1,60E-05 & 0,913 & 0,035 & 0,090 \\
		11 & 4,93E-05 & 1,146 & 0,035 & 0,221 \\
		13 & 2,01E-04 & 0,136 & 0,035 & 0,090 \\
		15 & 4,57E-03 & 0,126 & 0,035 & 0,327 \\
		\hline
	\end{tabular}
\caption{Tabela para a simulação sem ruído}
\end{table}


\begin{table}[H]
	\centering
	\begin{tabular}{|p{2.5cm}|p{2.5cm}|p{2.5cm}|}
		\hline
		Harmônico & MSE em frequência (\%) & MSE em amplitude (\%)\\
		\hline
		1  & 4,21E-05 & 2,32E-04 \\
		3  & 6,05E-07 & 3,16E-04 \\
		5  & 4,23E-07 & 5,42E-04 \\
		7  & 3,70E-07 & 1,01E-03 \\
		9  & 5,56E-07 & 1,45E-03 \\
		11 & 3,26E-07 & 1,96E-03 \\
		13 & 7,37E-07 & 3,65E-03 \\
		15 & 7,76E-07 & 4,56E-03 \\
		\hline
	\end{tabular}
	\caption{Tabela para a simulação com ruído ($\sigma ^2$=10)}
\end{table}

\begin{figure}[H]
	\centering    
	%	\def\svgwidth{\columnwidth}
	\def\svgscale{1}
	\input{images/harm15_ruido.pdf_tex}
	\caption{convergência do 15º harmônico na presença de ruído}
	\label{fig:esquema_pll}
\end{figure}

\begin{figure}[H]
	\centering    
	%	\def\svgwidth{\columnwidth}
	\def\svgscale{1}
	\input{images/harm15_deg.pdf_tex}
	\caption{convergência do 15º harmônico com degrau em frequência}
	\label{fig:esquema_pll}
\end{figure}

\subsection{inter-harmônicos}

Foram simuladas estimações com inter-harmônicos, seguindo as mesmas prescrições do caso anterior. Vemos que os tempos de convergência não são muito afetados, entretanto o erro quadrático médio para a fundamental aumenta um pouco. Isto se deve a um leve batimento que o harmônico 3.2 causa na fundamental, fazendo com que a estimação oscile um pouco. Entretanto se olhamos uma média grande o suficiente destas estimações, podemos nos livrar do batimento.

\begin{figure}[H]
	\centering    
	%	\def\svgwidth{\columnwidth}
	\def\svgscale{1}
	\input{images/batimento.pdf_tex}
	\caption{batimento observado na fundamental}
	\label{fig:esquema_pll}
\end{figure}

\begin{figure}[H]
	\centering    
	%	\def\svgwidth{\columnwidth}
	\def\svgscale{1}
	\input{images/harm64.pdf_tex}
	\caption{convergência do inter-harmônico 6.4}
	\label{fig:esquema_pll}
\end{figure}

\begin{table}[H]
	\centering
	\begin{tabular}{|p{2.5cm}|p{2.5cm}|p{2.5cm}|p{2.5cm}|p{2.5cm}|}
		\hline
		Harmônico & convergencia em freq. (s)& convergencia em amp (s) & MSE em frequência (\%) & MSE em amplitude (\%)\\
		\hline
		1    & 0,177083 & 0,1375   & 2,91E-05 & 1,66E-04 \\
		3,2  & 0,179167 & 0,175    & 6,73E-08 & 1,06E-05 \\
		6,4  & 0,114583 & 0,19375  & 2,46E-09 & 3,65E-06 \\
		11,3 & 0,11875  & 0,352083 & 5,04E-08 & 3,59E-06 \\
		\hline
	\end{tabular}
	\caption{Tabela para a simulação de inter-harmônicos}
\end{table}


\section{esforço computacional}

\indent Considerando um sinal amostrado em 7380  Hz e uma subamostragem de 16 amostras, será feita a estimativa do esforço computacional considerando operações realizadas a cada iteração de soma, multiplicação e funções trigonométricas, sem considerar deslocamentos de buffers e alocação de memória:

\subsection{Banco de filtros}

\indent Cada estrutura correspondente ao filtro apresentado na equação [\ref{eq:filtro}] representa 5 multiplicações e 4 somas. Sendo dois filtros, totaliza 10 multiplicações e 8 somas por cada amostra.

\subsection{PLL}

\indent Cada estrutura PLL como apresentada na equação [só dá pra citar quando juntar os capítulos] representa 11 multiplicações, 5 somas e 4 trigonométricas, que são executadas a cada 16 amostras. 

\subsection{Atualização de frequência}

\indent Cada atualização de frequência custa 2 multiplicações, uma soma e uma trigonométrica.

\subsection{Totais}

Abaixo temos uma tabela com todos os cálculos necessários para 1 segundo de sinal, por cada frequência rastreada. Percebemos que mais de 90\% dos cálculos são gastos pelo filtro digital.

\begin{table}[H]
	\centering
	\begin{tabular}{l|l|l|l}
		Estrutura   & Somas & Multiplicação & Trigonométricas \\
		\hline 
		Banco       & 61440 & 76800         & 0               \\
		PLL         & 2400  & 5280          & 1920            \\
		Atualização & 480   & 920           & 480             \\
		\hline
		Total       & 64320 & 83040         & 2400           
	\end{tabular}
\end{table}
