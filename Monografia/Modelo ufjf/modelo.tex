\documentclass[
        12pt,
        openany, %openright,			
        oneside, %twoside,			%% twoside: para frente e verso ao imprimir
        a4paper,			
        english,			
        brazil			        %% Idioma principal 
        ]{abntbibufjf}

\usepackage{lmodern}						
\usepackage[T1]{fontenc}		
\usepackage[utf8]{inputenc}		%% Para converter automaticamente acentos como digitados. Mude utf8 para latin1 se precisar. 
                                %% Permite digitar os acentos no teclado normalmente, sem comandos (\'e \`a , etc.).
\usepackage{lastpage}			
\usepackage{indentfirst}		
\usepackage{color}			
\usepackage{graphicx}			
\usepackage{microtype} 	

\usepackage{subfiles}
\usepackage{blindtext}

\usepackage[numbers]{natbib}
   

\usepackage{notoccite}
\usepackage{amsmath}


\usepackage{ wasysym } %só pro smiley
%\usepackage[num]{abntex2cite}   % Citações padrão ABNT
%\citebrackets[]


\usepackage{footnote}
\usepackage{url}
\usepackage{cancel}
\usepackage{bm}
\usepackage{epstopdf}
\usepackage{listings}
\lstset{
	literate=%
	{á}{{\'{a}}}1
	{í}{{\'{i}}}1
	{é}{{\'{e}}}1
	{ý}{{\'y}}1
	{ú}{{\'{u}}}1
	{ó}{{\'{o}}}1
	%{ě}{{\v{e}}}1
	{ç}{{\c{c}}}1
	{ã}{{\~{a}}}1
	{õ}{{\~{o}}}1
	{â}{{\^{a}}}1
	{ê}{{\^{e}}}1
	{š}{{\v{s}}}1
	{č}{{\v{c}}}1
	{ř}{{\v{r}}}1
	%{ž}{{\v{z}}}1
	{ď}{{\v{d}}}1
	{ť}{{\v{t}}}1
	{ň}{{\v{n}}}1                
	{ů}{{\r{u}}}1
	{Á}{{\'A}}1
	{Í}{{\'I}}1
	{É}{{\'E}}1
	{Ý}{{\'Y}}1
	%{Ú}{{\'U}}1
	{Ó}{{\'O}}1
	%{Ě}{{\v{E}}}1
	{Š}{{\v{S}}}1
	%{Č}{{\v{C}}}1
	{Ř}{{\v{R}}}1
	{Ž}{{\v{Z}}}1
	{Š}{{\v{S}}}1
	{š}{{\v{s}}}1
	{Ď}{{\v{D}}}1
	{Ť}{{\v{T}}}1
	{Ň}{{\v{N}}}1                
	{Ů}{{\r{U}}}1    
}


\usepackage{hyperref}
\usepackage{pdfpages}

%% -----------------------------------------------------------------------------

%% Obs.: Alguns acentos foram omitidos.

\titulo{Nome do TCC} %%Por exemplo, Titulo da tese
% \subtitulo{: subt\'itulo do trabalho}  %% Retirar o primeiro ``%'' desta linha se for utilizar subtitulo. Deixar os dois pontos antes, em ``: subt\'itulo'' . 
\autor{Tu Nombre}
\autorR{Gabriel Eduardo, de Lima Machado} %%Colocar o sobrenome do autor antes do primeiro nome do autor, separados por ,
\local{Juiz de Fora}
\data{2018} %%Alterar o ano se precisar
\orientador[Orientador:]{Marcelo Lima} %%Se precisar, troque [Orientador:] por [Orientadora:]
 \coorientador[Cooorientadora:]{Nome do coorientador } %% Retirar o primeiro ``%'' desta linha se tiver coorientador. Se precisar, troque por [Cooorientadora:]. 
\instituicao{Universidade Federal de Juiz de Fora}
\faculdade{Faculdade de Engenharia} %%Alterar, dentro de chaves {}, se precisar.
\programa{Engenharia Elétrica – Habilitação em Sistemas Eletrônicos} %%Alterar, dentro de chaves {}, se precisar.
\objeto{Trabalho de Conclusão de Curso}  %%Tese (Doutorado)
\natureza{Trabalho de Conclusão de Curso apresentado à Faculdade de Engenharia da Universidade Federal de Juiz de Fora, como requisito para obtenção do grau de Engenheiro Eletricista.} %%Trocar Matem\'atica por outro, se precisar.


%% Abaixo, prencher com os dados da parte final da ficha catalografica

\finalcatalog{1. Palavra-chave. 2. Palavra-chave. 3. Palavra-chave. I. Sobrenome, Nome do orientador, orient. II. Título.} %% Aqui fica 
% escrito a palavra ``T\'itulo'' mesmo, nao o do trabalho. Se tiver coorientador, os dados ficam depois dos dados 
%% do orientador (II. Sobrenome, Nome do coorientador, coorient.) e antes de ``II. T\'itulo'', o qual passa a ``III. T\'itulo''.

%% ---

\setlength{\parindent}{1.3cm}

\setlength{\parskip}{0.2cm}  

\setlength\afterchapskip{12pt}  


%% Iniciar o documento
\begin{document}

%% ELEMENTOS PRE-TEXTUAIS

%% Capa
\inserecapa

%% Folha de rosto
\inserefolhaderosto


%% Ficha catalografica. AO IMPRIMIR, DEIXAR NO VERSO DA FOLHA DE ROSTO.
\inserecatalog  


%% Folha de aprovacao
\begin{folhadeaprovacao}

  \begin{center}
    {\chapterfont \bfseries \insereautor}

    \vfill
    \begin{center}
      {\chapterfont\bfseries\inseretitulo \inseresubtitulo}
    \end{center}
    \vfill
    
    \hspace{.45\textwidth}
    \begin{minipage}{.5\textwidth}
        \inserenatureza
    \end{minipage}%
    \vfill
  \end{center}
        
  Aprovado em: %%COLOCAR A DATA 
   
  \begin{center} BANCA EXAMINADORA \end{center}
  \assinatura{Prof. Dr. \insereorientador \ - Orientador \\ Universidade Federal de Juiz de Fora} 
%  \assinatura{Professor Dr. \inserecoorientador \ - Coorientador \\ Universidade Federal de Juiz de Fora}
  \assinatura{Professor Dr. ??? \\  Universidade Federal de Juiz de Fora}
  \assinatura{Professor  Dr. ??? \\ Universidade Federal de Juiz de Fora} 
%  \assinatura{...} %%RETIRE O % E PREENCHA SE PRECISAR
%  \assinatura{...}
%  \assinatura{...}

% \includepdf{ata.pdf} depois deverá ser substituído pela ata de defesa assinada

\end{folhadeaprovacao}


%% Dedicatoria. OPCIONAL. Retirar o ``%'' de cada das 4 linhas abaixo, caso queira.
% \begin{dedicatoria} \vspace*{\fill} \centering \noindent
%   \textit{ Dedico este trabalho ... (opcional)} 
%   \vspace*{\fill}
% \end{dedicatoria}


%% Agradecimentos. OPCIONAL. CASO SEJA BOLSISTA, INSERIR OS DEVIDOS AGRADECIMENTOS.
\begin{agradecimentos}

De acordo com a Associa\c{c}\~ao Brasileira de Normas T\'ecnicas - 14724 (2011, p. 1) Agradecimentos 
\'e o ``texto em que o autor faz agradecimentos dirigidos \`aqueles que contribu\'iram de maneira relevante \`a elabora\c{c}\~ao do trabalho.'' 
  



\end{agradecimentos}

%% Epigrafe. OPCIONAL
\begin{epigrafe}
    \vspace*{\fill}
	\begin{flushright}
		``Um orientador em paz não quer guerra com ninguém''\\
		(Exuperry Barros Costa)
	\end{flushright}
\end{epigrafe}


%% RESUMOS

%% Resumo em Portugu^es. OBRIGATORIO.
\setlength{\absparsep}{18pt} 
\begin{resumo}

De acordo com a Associa\c{c}\~ao Brasileira de Normas T\'ecnicas - 6028 (2003, p. 2) ``o resumo deve ressaltar 
o objetivo, m\'etodo e as conclus\~oes do documento (...) Deve ser composto de uma sequ\^encia de frases 
concisas, afirmativas e n\~ao de enumera\c{c}\~ao de t\'opicos. Recomenda-se o uso de par\'agrafo \'unico.''
O resumo deve ter de 150 a 500 palavras.

Palavras-chave: Palavra-chave. Palavra-chave. Palavra-chave.
%finalizadas por ponto e inicializadas por letra maiuscula.

\end{resumo}
 
 
%% Resumo em Ingle^s
\begin{resumo}[ABSTRACT]
  \begin{otherlanguage*}{english}
   ...

Key-words: ...
 \end{otherlanguage*}
\end{resumo}

%% Seguindo o mesmo modelo acima, pode-se inserir resumos em outras linguas. 

%% Lista de ilustracoes. OPCIONAL.
\pdfbookmark[0]{\listfigurename}{lof}
\listoffigures*
\cleardoublepage


% Lista de tabelas. OPCIONAL. Retire o ``%'' de cada das 3 linhas seguintes, caso queira.
\pdfbookmark[0]{\listtablename}{lot}
\listoftables*
\cleardoublepage

% Lista de abreviaturas e siglas. OPCIONAL
\begin{siglas} %%ALTERAR OS EXEMPLOS ABAIXO, CONFORME A NECESSIDADE
 \item[ABNT] Associa\c{c}\~ao Brasileira de Normas T\'ecnicas
 \item[UFJF] Universidade Federal de Juiz de Fora
 \item[IBGE] Instituto Brasileiro de Geografia e Estat\'istica
\end{siglas}

%% Lista de simbolos. OPCIONAL
\begin{simbolos} %%ALTERAR OS EXEMPLOS ABAIXO, CONFORME A NECESSIDADE
  \item[$ \forall $] Para todo
  \item[$ \in $] Pertence
  \item[$\smiley$] Sorrino
  \item[$\frownie{}$] Chorano
  \item[$\theta_1$] Theta 1

 \end{simbolos}

 
%% Sumario
\pdfbookmark[0]{\contentsname}{toc}
\tableofcontents*
\cleardoublepage

%% ----------------------------------------------------------

%% ELEMENTOS TEXTUAIS

\textual
\pagestyle{simple}   




\chapter{INTRODUÇÃO}  

\section{planejamento}


\chapter{CONCLUSÕES E TRABALHOS FUTUROS}

\section{CONCLUSÕES}


%% ----------------------------------------------------------


%% ELEMENTOS POS-TEXTUAIS

\postextual

%referências
% \bibliographystyle{abnt-num}
 \bibliographystyle{IEEEtranN}
\bibliography{bibliografia}


%% Apendices

%%\begin{apendicesenv}

%%\chapter{}

%%\begin{verbatim}

%%\end{verbatim}

%%\end{apendicesenv}

%% Anexos


%%% ---
\end{document}
